%\documentclass[12pt, letterpaper, titlepage]{article}
\documentclass[12pt, letterpaper]{article}

\usepackage{amsmath, amsfonts}
\usepackage{booktabs}
\usepackage{amsthm}
\usepackage{graphicx}
\usepackage[margin=1in]{geometry}
\usepackage{hyperref}
\usepackage{cleveref}
\hypersetup{colorlinks = true, linkcolor = blue, citecolor=blue, urlcolor = blue}
\usepackage{natbib}
\usepackage{float}
\usepackage{setspace}
\usepackage{pdfpages}
\usepackage{lineno}
\usepackage{mwe}
\usepackage{comment}
\linenumbers*[1]
% %% patches to make lineno work better with amsmath
\newcommand*\patchAmsMathEnvironmentForLineno[1]{%
 \expandafter\let\csname old#1\expandafter\endcsname\csname #1\endcsname
 \expandafter\let\csname oldend#1\expandafter\endcsname\csname end#1\endcsname
 \renewenvironment{#1}%
 {\linenomath\csname old#1\endcsname}%
 {\csname oldend#1\endcsname\endlinenomath}}%
\newcommand*\patchBothAmsMathEnvironmentsForLineno[1]{%
 \patchAmsMathEnvironmentForLineno{#1}%
 \patchAmsMathEnvironmentForLineno{#1*}}%

\AtBeginDocument{%
 \patchBothAmsMathEnvironmentsForLineno{equation}%
 \patchBothAmsMathEnvironmentsForLineno{align}%
 \patchBothAmsMathEnvironmentsForLineno{flalign}%
 \patchBothAmsMathEnvironmentsForLineno{alignat}%
 \patchBothAmsMathEnvironmentsForLineno{gather}%
 \patchBothAmsMathEnvironmentsForLineno{multline}%
}

% control floats
\renewcommand\floatpagefraction{.9}
\renewcommand\topfraction{.9}
\renewcommand\bottomfraction{.9}
\renewcommand\textfraction{.1}
\setcounter{totalnumber}{50}
\setcounter{topnumber}{50}
\setcounter{bottomnumber}{50}

\newcommand{\jy}[1]{\textcolor{blue}{JY: #1}}
\newcommand{\eds}[1]{\textcolor{red}{EDS: (#1)}}
\newcommand{\of}[1]{\textcolor{violet}{OF: #1}}

% NOTE: To produce blinded version, replace "0" with "1" below.
\newcommand{\blind}{0}

%\title{On Devon Allen's Disqualification at the 2022 World Track and Field
%Championships}
%
%\author{Owen Fiore\\
%%   \href{mailto:owen.fiore@uconn.edu}
%% {\nolinkurl{owen.fiore@uconn.edu}}\\
  %Elizabeth D. Schifano\\
  %Jun Yan\\[1ex]
  %Department of Statistics, University of Connecticut\\
%}
%\date{}

\begin{document}
%\maketitle

\if0\blind
{
  \title{\bf On Devon Allen's Disqualification at the 2022 World Track and Field
Championships}
  \author{Owen Fiore, %\\
%   \href{mailto:owen.fiore@uconn.edu}
% {\nolinkurl{owen.fiore@uconn.edu}}\\
  Elizabeth D. Schifano, %\\
  Jun Yan\\[1ex]
  Department of Statistics, University of Connecticut\\
}
\date{}
  \maketitle
} \fi

\if1\blind
{
  \bigskip
  \bigskip
  \bigskip
  \begin{center}
  {\LARGE\bf On Devon Allen's Disqualification at the 2022 World Track and Field
Championships}
\end{center}
  \bigskip
} \fi


\doublespace

% Abstract should be under 200 words

\begin{abstract}
Devon Allen’s disqualification at the men's 110-meter hurdle final at
the 2022 World Track and Field Championships, 
due to a reaction time (RT) of 0.099 seconds---just 0.001 seconds below
the allowable threshold---sparked widespread debate over 
the fairness and validity of RT rules. This study investigates two key
issues: variations in timing systems and the justification for the
0.1-second disqualification threshold. We
pooled RT data from men’s 110-meter hurdles and 100-meter dash, as
well as women’s 100-meter hurdles and 100-meter dash, spanning
national and international competitions. Using a rank-sum test for
clustered data, we compared RTs across multiple competitions,
while a generalized Gamma model with random effects for venue and heat
was applied to evaluate the threshold. Our analyses reveal significant
differences in RTs between the 2022 World Championships and other
competitions, pointing to systematic variations in timing
systems. Additionally, the model shows that RTs below 0.1 seconds,
though rare, are physiologically plausible. These findings highlight
the need for standardized timing protocols and a re-evaluation of the
0.1-second disqualification threshold to promote fairness in
elite competition.


\bigskip\noindent{\sc Keywords}:
false start, GAMLSS, reaction time, rank-based test, short sprint
\end{abstract}

\doublespace


\section{Introduction}
\label{sec:intro}

Devon Allen’s highly anticipated performance at the 2022 World
Track and Field Championships in Eugene, Oregon, ended in
controversy when he was disqualified for a reaction time (RT) of
0.099 seconds, just 0.001 seconds below the allowable threshold.
Allen, a University of Oregon alumnus,
had recently run a time of 12.84 seconds in the 110-meter hurdle
event, just 0.04 seconds short of the world record. After placing
third at the U.S. Track and Field Championships, he advanced
through the preliminary heats and semifinals at the World
Championships, with RTs of 0.123 and 0.101 seconds,
respectively. However, in the final heat, competing in front of his
home audience, Allen’s RT was just 0.001 seconds faster
than the 0.1-second threshold set by the International Association
of Athletics Federations (IAAF), resulting in his disqualification, a
decision that was met with widespread public outcry. This 
incident highlighted two long-standing issues: variability in the
measurement of RTs by Start Information Systems (SIS) and the
appropriateness of the 0.1-second disqualification threshold. As
reaction times are measured in fractions of a second, inconsistencies
in timing technologies and rules can significantly affect athlete
outcomes, raising questions about fairness and standardization.


World Athletics (formerly IAAF) uses certified SIS to measure RTs, yet
variation in technology persist. Discussions at online forums such as
\url{www.LetsRun.com} questioned consistencies in the SIS as a
contributing factor to RT anormalies \citep{johnson2022data,
  johnson2022was}. Historically, ``loud gun'' systems caused signal delays
for athletes in outer lanes  due to the speed of sound, an issue
addressed with the introduction of ``silent gun'' systems in 2010,
which electronically synchronize sound delivery to all athletes
\citep{tonnessen2013reaction}. Despite these advances, variability
persists due to differences in sensor technologies, such as force
transducers and accelerometers, and inconsistencies in event detection
algorithms \citep{willwacher2013novel}. For example, simple
force-threshold systems may delay RT detection by up to 26~ms compared
to more sophisticated methods \citep{pain2007sprint}. These findings
emphasize the need for standardized certification protocols to reduce
discrepancies and ensure fairness in RT measurements, as recently
reviewed by \citet{milloz2021sprint}.


The 0.1-second disqualification threshold, introduced in the 1990s,
has been the subject of significant debate. Partly based on limited
data from Finnish national-level athletes~\cite{mero1990reaction}, the
threshold may not adequately represent the capabilities of elite
sprinters. Controlled experiments have shown that reaction times below
0.1 seconds are physiologically plausible \citep{pain2007sprint,
  komi2009iaaf}, while retrospective analyses of competition data
often advocate for raising the threshold~\cite{brosnan2017effects,
  lipps2011implications}. Stricter false-start rules, introduced to
minimize race disruptions, have also discouraged sprinters from
attempting faster starts, which may artificially inflate RTs recorded
in competition~\cite{haugen2013effect}. This divergence between
experimental findings and competition-based analyses illustrates the
complexity of defining a universally fair threshold. Addressing these
debates requires modern data collection and advanced methodologies to
ensure equity and consistency in elite
competition~\cite{milloz2021sprint}.


This paper addresses two primary objectives from a statistical
perspective, using modern methodologies to analyze historical
data. First, we investigate whether reaction times (RTs) at the 2022
World Championships were significantly different from other
competitions, focusing on athletes who competed in multiple
events. Using a matched-pairs design, we compare RTs across the 2022,
2019, and 2023 World Championships, as well as the 2022 national-level
competitions. This approach isolates the effect of the competition
year while controlling for individual performance. The clustered
nature of the data, with the goal of assessing differences across
events within athletes, was analyzed using a rank-based comparison
approach for clustered data \citep{datta2005rank}. Second, we evaluate
the appropriateness of the 0.1-second disqualification threshold by
modeling RTs from World Championships held from 1999 onward. A
generalized Gamma (GG) distribution with random effects for both venue and
heat was applied within the framework of the generalized additive
model for location, scale, and shape (GAMLSS)
\citep{rigby2005generalized, stasinopoulos2024generalized}.
This model enables estimation of the
probability of RTs falling below the threshold, providing a
statistical assessment of RT consistency and the validity of the
current threshold.


The rest of this paper is organized as follows. Section~\ref{sec:Data}
describes the data collection and datasets used for the two
objectives. Section~\ref{sec:rank} outlines the rank-based comparison
methods for objective~1, rank-based comparison for clustered data with
subunit-level gruping to analyze differences in RTs across different
competitions. Section~\ref{sec:gamlss} details the methods for
objective~2, the GG model with random effects to evaluate the appropriatenes
of the 0.1-second threshold. Section~\ref{sec:Results} presents the
results of both objectives and discusses the implications.
Finally, Section~\ref{sec:concludingremarks} highlights the
paper’s impact, limitations, and potential for future research.
All data and code for our analysis are provided in the supplementary
materials.



\section{Data}
\label{sec:Data}


This study focuses on RTs from sprint events where reaction
time plays a critical role. Specifically, we included data from men’s
100-meter dashes and 110-meter hurdles, as well as women’s 100-meter
dashes and 100-meter hurdles. Statistical tests showed no significant
differences in RTs between these events, supporting their
inclusion in a unified analysis. However, we excluded data from
200-meter dashes and longer events, as their RTs were found
to be significantly different. This difference is not unexpected, as
RT is generally less critical in longer sprint events
compared to shorter distances where explosive starts are paramount.


Two datasets were used to address the study's objectives. To
investigate whether RTs at the 2022 World Championships
were significantly different from other competitions (Objective 1), we
used data from male athletes in the 110-meter hurdles and 100-meter
dash, and female athletes in the 100-meter hurdles and 100-meter dash,
provided they competed in the 2022 World Championships and at least
one other competition (2022 national championships, 2019 World
Championships, or 2023 World Championships). To assess the
appropriateness of the 0.1-second RT threshold (Objective
2), we used RTs from the 110-meter hurdles and 100-meter
dash across all World Championships held from 1999 to 2023. This
investigation began shortly after the 2022 World Championships, and we
are pleased to include data from 2023, which did not significantly
alter our results \citep{WAData}.


\subsection{Comparison Data with 2022 World Championships}
\label{sec:databeyond}

For assessing the abnormality of the 2022 World Championship,
the motivation for including data beyond the World Championships arose from
an exploratory analysis of how United States athletes performed at the 2022
United States Track and Field Championships, held from June 23--26, 2022, at
Hayward Field in Eugene, Oregon. Since the same venue hosted the 2022 World
Championships in August, this provided a unique opportunity to establish a
baseline for RTs.


\subsubsection{2022 National Competitions}
\label{sec:datanational}


\jy{Owen, are there names from 100-m dash?}
\of{Yes, there were four US athletes in the 100m dash that we have data on.
Marvin Bracy, Fred Kerley, Travyon Bromwell, and Christian Coleman.  All of whom
recorded their slowest times at the US Track and Field Championship.}
To compare RTs at the 2022 World Championships with those
recorded at national competitions, we constructed a dataset of athletes
who competed in both settings. Initially, we examined data from four
United States athletes---Trey Cunningham, Daniel Roberts, Grant
Holloway, and Devon Allen---who participated in both the United States
Track and Field Championships and the World Championships. Each athlete
recorded faster RTs in all World Championships races compared
to their performances at the national-level event. However, this small
sample size was insufficient for meaningful statistical analysis. To
expand the dataset, we included RTs from additional national
competitions held between May and July 2022 across various countries.
Compiling this data was challenging, as RTs were not centrally
located and often required searching country-specific websites, with many
results recorded in native languages.


The final dataset consisted of RTs from athletes who competed
in both national and international competitions. RTs from
preliminary heats, semifinals, and finals were included to ensure
sufficient data for analysis. Excluding preliminary heats would have
significantly reduced the number of athletes and clusters. Each
athlete was considered as a cluster, with observed RTs
from both the `treatment' group (2022 World Championships) and
the `control' group (national competition) within the cluster. Cluster
sizes ranged from three to six, with a median size of four. Because gender
is known to influence RTs \citep{babicc2009reaction,
  lipps2011implications}, we prepared data for men and women
separately, resulting in 80 RTs from 17 athletes for each
gender. This setup forms the basis for the rank-based comparison for
clustered data with subunit-level grouping, which accounts for the
within-athlete dependence inherent in this data structure. The top
panel of Figure!~\ref{fig:RankScatterplots} shows the RTs
of those who competed both at national competitions and the 2022 World
Championships.


\jy{Could you do a pooled analysis since there is no gender difference
  and event differece in the supplement?}
\of{Asked you about this in an email.}



\begin{figure}[tbp]
  \centering
  \includegraphics[width=\textwidth]{RankScatterPlots}
  \caption{RTs for athletes who competed at the 2022 World Championships
	and at another championship (2022 national, 2019 World, or 2023 World) at which
	they competed. On the horizontal
  axis below each graph "H", "S", and "F" refer to the heats, semifinals, and
  finals respectively. Please note that in the last row the 2022 times are to
  the right of the 2023 times.
	\eds{Please check my edited caption; also, I suggest rearranging the rows to
	use the same ordering as the data was described: put national vs international
	in first row, 2019 vs 2022 in second row, and 2022 vs 2023 in third row}
	}
  \of{I agree and I updated it. I think it looks good now.}
  \label{fig:RankScatterplots}
\end{figure}

\subsubsection{2019 and 2023 World Championships}
\label{sec:data2019}

To compare RTs at the 2022 World Championships with those
recorded in the 2019 and 2023 World Championships, we prepared datasets
including RTs from athletes who competed in the 2022 World
Championships and at least one of the 2019 or 2023 World Championships.
Specifically, we identified athletes who participated in both the 2019
and 2022 World Championships, as well as those who participated in both
the 2022 and 2023 World Championships. In the 2019-2022 comparison,
RTs from 2022 were treated as the `treatment' group, with
2019 serving as the `control' group. Similarly, in the 2022-2023
comparison, RTs from 2022 were treated as the `treatment'
group, with 2023 serving as the `control' group. This structure allowed
us to prepare datasets suitable for examining RTs of athletes
who competed across multiple World Championships.


Each athlete was treated as a single cluster, containing their reaction
times from different World Championships. The dataset for the 2019-2022
comparison contained 134 RTs from 34 male athletes and 124
RTs from 31 female athletes. The dataset for the 2022-2023
comparison contained 161 RTs from 45 male athletes and 182
RTs from 47 female athletes. While it is theoretically
possible that athletes improved their RTs between 2019 and
2022 or between 2022 and 2023, such improvements are highly unlikely for
elite sprinters, as they already operate near the limits of human
performance. Consequently, consistent improvements observed in 2022
would suggest systematic differences rather than natural variability.


Figure~\ref{fig:RankScatterplots} shows the RTs of athletes
who competed in both the 2019 and 2022 World Championships (middle
panel) and those who competed in both the 2022 and 2023 World
Championships (lower panel). The athletes included in the 2022 data
differ between these two comparisons, as the set of athletes who
competed in both 2019 and 2022 is not the same as the set who competed
in both 2022 and 2023. To be included, athletes must have competed in
at least one race at each championship. Notably, Devon Allen recorded
the fastest RTs in both the Finals and Semifinals of the
2022 World Championships, but his disqualification was determined by a
difference of just 0.002 seconds, with RTs of 0.101 and
0.099 seconds, respectively. This highlights the critical role of
RT precision in elite-level competition.

\eds{I also think we need to comment on the men and women's data. But within row
of the figure, don't the distributions appear similar?  If that is so, would a
reviewer ask why we didn't pool the data for men and women?}
\jy{Let's put the pooled analysis in the supplement.}
\of{Please see my email}

\subsection{Data from World Championships 1999--2023}
\label{sec:dataworld}

The data for evaluating the appropriateness of the 0.1-second
threshold was obtained from World Athletics and covers the men's
110-meter hurdles and 100-meter dashes from 1999 to 2023. Due to
possible gender differences \citep{babicc2009reaction,
  lipps2011implications}, data for women's 100-meter hurdles and
100-meter dashes and their analyses were relegated to the
supplementary material. We focus on the RTs recorded during
semifinal and final heats only, as RTs from preliminary
heats are often not as fast as those in later heats
\citep[e.g.,][]{collet1999strategic, tonnessen2013reaction,
  brosnan2017effects, zhang2021correlation}. For analysis purposes, we
pooled RTs from semifinal and final heats to increase sample
size, which is particularly important for years with limited final heat
observations. For example, in 2022, only five data points were available
from the final heat due to two disqualifications and one athlete not
competing. Unless otherwise noted, this pooled dataset forms the basis
for our analysis throughout the paper. Additionally, we consider datasets
that exclude 2022 to assess how our findings might differ when excluding
this year of interest.


\jy{have the numbers in this paragraph been updated with the
  additional 100-dash data?}
  \of{Funny enough the median reaction time remained exactly the same.}
The data is summarized in Figure~\ref{fig:Boxplot}, which presents a
sequence of boxplots of RTs from 1999 to 2023. It is evident
that RTs in 2022 were notably faster, with a median reaction
time of 0.129 seconds compared to the 0.156 seconds observed in earlier
studies, such as \citet{brosnan2017effects} for data spanning 1999 to 2014.
Figure~\ref{fig:Boxplot} also highlights year-to-year variability in
RTs, likely influenced by changes in the championship venue
and environmental conditions such as humidity, precipitation, and
elevation. Furthermore, advancements in technology and alterations to
false start rules during the study period may have played a role in these
variations \citep{willwacher2013novel}.


\begin{figure}[tbp]
  \centering
  \includegraphics[width=\textwidth]{Boxplot}
  \caption{The RTs from 1999 to 2023 for the men's 110 meter hurdle
  and 100 meter dash.}
  \label{fig:Boxplot}
\end{figure}


Between 2007 and 2009, World Athletics allowed one
false start warning before disqualifying a sprinter \citep{iaaf2009falsestart}.
This lenient rule led to 18 male and 7 female false starts at both the 2007
and 2009 World Championships. In 2011, this rule was replaced with the stricter
policy of automatic disqualification for false starts, aimed at reducing the
delays caused by repeated warnings. This change reduced men’s false starts
by two-thirds in 2011, with only six male and four female disqualifications
\citep{iaaf2009falsestart}. \citet{haugen2013effect} demonstrated that more
lenient false start rules significantly improved RTs during the
1997–2009 period, suggesting that rule changes over the study period may
have contributed to variations in RTs across years.


\section{Methods} \label{sec:methods}

The data described in Sections~\ref{sec:databeyond} and~\ref{sec:dataworld} were
analyzed with rank-based methods and a GAMLSS, respectively.


\subsection{Rank-based Comparison}\label{sec:rank}


To test the conjecture that the 2022 World Championships timing device may have
led to faster recorded RTs, we compare the RTs of the same
athletes who have attended both the 2022 World Championships and other
competitions.
In this setting, we have clustered data with subunit grouping. In particular,
each athlete is a cluster and the multiple RTs from the same athlete
can be from either the 2022 World Championships or otherwise.
Let $X_{ij}$ be the $j$th RT of athlete~$i$, $i = 1, \ldots, n$,
$j = 1, \ldots, m_i$ where $m_i$ is the number of observations from
athlete~$i$. Let $\delta_{ij}$ be the group indicator of $X_{ij}$; $\delta_{ij}
= 1$ if $X_{ij}$ is in group~1 (2022 World Championships) and $\delta_{ij} = 0$
otherwise. Athletes are
assumed to be independent, while subunit observations from the same athlete are
not. The null hypothesis $H_0$ to be tested is that there is no difference
between the two groups; i.e., the distribution of $X_{ij}$ remains the same
regardless of the group indicator $\delta_{ij}$.


\citet{datta2005rank} proposed an extension of the Wilcoxon rank-sum test to
clustered data with subunit-level grouping. The test is designed based on a
within-cluster resampling principle. Consider randomly picking one observation
from each cluster to form a pseudo-sample. Let $X_i^*$ be a random pick from the
$i$th cluster in the pseudo-sample and $\delta_i^*$ its group indicator. The
Wilcoxon rank-sum statistic for the pseudo-sample is
\[
W^* = \frac{1}{n + 1} + \sum_{i=1}^{n} \delta_{i}^{*} R_{i}^{*},
\]
where $R_{i}^{*}$ is the rank of $X_{i}^{*}$ in the pseudo-sample.
The test statistic $S$ is the average of $W^*$ averaged over all possible
pseudo-samples conditioning on the observed data and group indicators.
The mean and variance of $S$ under $H_0$ can be derived so that $S$ can be
standardized to form a $Z$ statistic which follows a standard normal distribution
asymptotically \citep[p.910]{datta2005rank}.


When the sample size is small, the asymptotic normal distribution may
not be reliable, so we also use 1~million random permutations to
simulate the null distribution of the test statistic.
This method is available from the \texttt{clusWilcox.test()} function
with \texttt{method = `ds'} (for \underline{D}atta and \underline{S}atten) and
\texttt{exact = TRUE} from R package
\texttt{clusrank} \citep{jiang2020wilcoxon}.


\subsection{GAMLSS}\label{sec:gamlss}

Based on an exploratory analysis, the RTs are adequately
modeled by a GG distribution with random effects in
model parameters. The GG distribution has three parameters, denoted by
$\text{GG}(\mu, \sigma, \nu)$ has density function
\begin{equation}
  \label{eq:gg}
f_Y(y \mid \mu, \sigma, \nu) =
\frac{|\nu| \theta^\theta z^{\theta}}{\Gamma(\theta) y}
\exp\left(-z \theta\right),
\end{equation}
for $y > 0$, $\mu > 0$, $\sigma > 0$, and $\nu \neq 0$,
where $z = (y / \mu)^\nu$,
$\theta = 1 / (\sigma^2 \nu^2)$, and
$\Gamma(\cdot)$ denotes the Gamma function.
The GG distribution is highly flexible, encompassing several
well-known distributions as special cases, such as the
Weibull ($\mu = \nu)$ and  Gamma $(\nu = 1)$ distributions.
Its expectation is
\[
  \frac{\mu \Gamma(\theta + 1 / \nu)}
  {\theta^{1 / \nu} \Gamma(\theta)},
\]
provided $\theta > -1 / \nu$. Here,
$\mu$ scales the central tendency, $\sigma$ controls
dispersion, and $\nu$ determines skewness. This parameterization allows
the distribution to model asymmetric and heavy-tailed data effectively, making
it particularly suitable for RTs.
An implementation of this distribution is available from R package
\texttt{gamlss.dist} \citep{rigby2019distributions}.


Let $Y_{ijk}$ denote the RT of observation~$k$ in heat~$j$
of year~$i$. Conditioning on a venue effect $v_i$ for year~$i$
and a heat effect $h_{i/j}$ nested within each year~$i$, the
distribution of $Y_{ijk}$ is
$\text{GG}(\mu_{ijk}, \sigma_{ijk}, \nu)$, where
\begin{align}
\log(\mu_{ijk}) &= \beta_0 + v_i , \label{eq:mu}\\
\log(\sigma_{ijk}) &= \gamma_0 + h_{i/j} , \label{eq:sigma}
\end{align}
$v_i$ is normally distributed with mean zero and
variance~$\sigma_v^2$, and $h_{i/j}$ is normally distributed with mean
zero and variance~$\sigma_h^2$.
The two random effects were found useful: one capturing the venue effect, which
is used to contrast years, and the second being the heat effect, where every
race was given a unique identifier with typically five to nine observations
per race. The heat effect is important as it captures the variability in the
amount of time athletes are on the starting blocks before the gun goes off.
This model can be fit with R package \texttt{gamlss}
\citep{stasinopoulos2008generalized}.


Model diagnosis and tail analysis can be done with the fitted GG model
from package \texttt{gamlss}. Normalized quantile residuals, or
z-scores \citep{dunn1996randomized}, of the observations can be
extracted with the \texttt{residuals} method of a \texttt{gamlss}
object. The z-scores can then be checked with a Q-Q plot
\citep{almeida2018ggplot2}. The marginal
distribution of $Y_{ijk}$ is a scale-mixture of GG distributions, which can be
easily simulated from once the parameters are estimated. Many
random numbers generated from the fitted mixture distribution can be used to
approximate the probability of observing a RT faster than any given
threshold. We are specifically interested in the probability of a RT
being less than 0.1 seconds in order to gauge if that is a reasonable
disqualification barrier.



\section{Results} \label{sec:Results}

\subsection{Rank-based Comparison} \label{subsec:Results_Rank}

The rank-based methods described in Section~\ref{sec:rank} were used to
compare RTs between the 2022 World Championships and other
competitions in which the same athletes participated. These comparisons
were conducted separately for men and women, resulting in six total
comparisons: RTs from the 2022 national-level championships
versus the 2022 World Championships for men and women, RTs
from the 2019 versus 2022 World Championships for men and women, and
RTs from the 2022 versus 2023 World Championships for men and
women.


\begin{table}
  \centering
  \caption{P-values of comparisons between
    RTs from different competitions for the same athletes.
    2022 Nat. vs Inter. compares RTs from 2022 national-level
    championships and the 2022 World Track and Field Championships. 2019
    vs 2022 compares RTs from the 2019 and 2022 World Track and
    Field Championships. 2022 vs 2023 compares RTs from the
    2022 and 2023 World Track and Field Championships.}
  \begin{tabular}{c c c c c}
   \toprule
   Comparison & Permutation & Asymptotic & \# of athletes & \# of observations  \\
   \midrule
   2022 Nat. vs Inter. Men & $1.0 \cdot 10^{-6}$ & $ 6.1 \cdot 10^{-5}$ & 17 & 80 \\
   2022 Nat. vs Inter. Women & $1.0 \cdot 10^{-6}$ & $ 1.2 \cdot 10^{-3}$ & 17 & 80 \\[1ex]
   2019 vs 2022 Men & $2.8 \cdot 10^{-5}$ & $1.1 \cdot 10^{-5}$ & 34 & 134 \\
   2019 vs 2022 Female & $ 1.5 \cdot 10^{-3}$ & $6.9 \cdot 10^{-3}$ & 31 & 124 \\[1ex]
   2022 vs 2023 Men & $1.0 \cdot 10^{-6}$ & $1.4 \cdot 10^{-6}$ & 45 & 161 \\
   2022 vs 2023 Female & $1.0 \cdot 10^{-6}$ & $9.4 \cdot 10^{-7}$ & 47 & 182 \\
   \bottomrule
  \end{tabular}
  \label{tab:Clusrankresults}
\end{table}



Table~\ref{tab:Clusrankresults} presents the results from both permutation
and asymptotic rank-based tests for these six comparisons. The tests reveal
that the national versus international comparisons yielded the lowest
p-values, indicating stronger evidence of faster RTs at the
2022 World Championships compared to the 2022 national competitions. For
both men and women, permutation test results were highly significant,
providing substantial evidence that athletes competing at the 2022 World
Championships had significantly faster RTs than they did just
a few months earlier at national-level events. The comparisons between the
2019 and 2022 World Championships and between the 2022 and 2023 World
Championships also produced significant results. Collectively,
these results suggest that athletes consistently achieved faster reaction
times at the 2022 World Championships compared to other competitions. This
finding supports the hypothesis that the 2022 World Championships presented
conditions conducive to faster RTs, potentially due to systematic
or environmental factors.


\subsection{GAMLSS} \label{subsec:Results_GLMM}

The fitted parameters of the GG distribution int he
GAMLSS framework in Equations~\eqref{eq:gg}--\eqref{eq:sigma} are
summarized in Table~\ref{tab:ggfit}. Results obtained from both
excluding and including 2022 data are reported. The fixed-effect
parameters include $\beta_0$, $\gamma_0$, and $\nu$, corresponding to
the intercept of the log-location, log-scale, and shape of the GG
distribution, respectively. Random effects account for variability at
the venue level $\sigma_v$ on the log-scale of the $\mu$ parameter
and at the heat level $\sigma_h$ on the log-scale of the $\sigma$
parameter in the density in Equation~\eqref{eq:gg}. The variance
of the venue-level random effect is smaller than the heat-level random
effect variance, suggesting that heat-level variability in the scale
parameter is substantial, though on the dispersion parameter. When the
2022 data is included, the fixed-effect estimates remain stable,
indicating robustness, but the venue-level random-effect variance
increases to 0.0576, reflecting additional variability in the location
parameter, while the heat-level random-effect variance remains stable.
decreases slightly to 0.3198. These results
highlight that RTs are influenced by both venue and heat-level
factors, with heat-level variability playing a more prominent role in the scale
parameter, and that the inclusion of 2022 introduces greater venue-level
variability, likely due to systematic differences in RTs that year.


\jy{Please add standard errors, which could be from \texttt{summary(gg3b)}}

\jy{Are these numbers correct? I modified with the gg3b in gg.R}
\of{Yes I agree with your numbers and have included the ones for the other model}

\of{I moved the table down because in the pdf it was appearing in the rank
section.} 
\of{I think it's confusing that sigma appears as a parameter of the model
and we use it as the standard deviation of the heat effect which appears in the
sigma parameter. I don't know much about statistical nomenclature in papers, is
this common?}

\begin{table}
  \centering
  \caption{Estimated fixed-effect parameters with standard errors in
    parentheses and estimated variance of the random effects from the
    fitted GG distribution with venue level random
    effects in $\mu$ and heat level random effects in $\sigma$ in
    Models~\eqref{eq:gg}--\eqref{eq:sigma}.}
  \label{tab:ggfit}
  \begin{tabular}{c c c c c c}
    \toprule
    Data set & $\beta_0$ & $\gamma_0$ & $\nu$ & $\sigma_v$ & $\sigma_h$ \\
    \midrule
    Excluding 2022 & $-$1.910 (0.005) & $-$2.200 (0.025) & $-$1.177 (0.442) & 0.043 & 0.326 \\
    Including 2022 & $-$1.910 (0.005) & $-$2.200 (0.027) & $-$1.178 (0.447) & 0.058 & 0.320 \\
    \bottomrule
  \end{tabular}
\end{table}

\begin{figure}[tbp]
  \centering
  \includegraphics[width=\textwidth]{diagnosis.pdf}
  \caption{Diagnosis of the fitted GG distribution with
    random effects in model parameters: kernerl density of 1 million
    observations drawn from the fitted model overlaid with the
    histogram of the observed RTs (left); Q-Q plot of the
    normal z-score of the quantile residuals from the fitted model.}
  \label{fig:diagnosis}
\end{figure}


Figure~\ref{fig:diagnosis} presents diagnostic checks for the fitted
GG distribution model with random effects. The left
panel compares the kernel density estimate of one million simulated reaction
times from the fitted model to the histogram of the observed RTs.
The close alignment between the density curve and the histogram suggests that
the fitted GG model adequately captures the overall distribution of the
RTs. The right panel shows a Q-Q plot of the z-scores of the
quantile residuals from the fitted model. The points lie approximately along
the 45-degree reference line, indicating that the residuals are consistent
with the standard normal distribution, supporting the adequacy of the model
fit. These diagnostics collectively demonstrate that the fitted model provides
a reasonable representation of the observed RT data.


% \begin{figure}[tbp]
%   \centering
%   \includegraphics{ComparisonOfVenueEffects}
%   \caption{The venue effects from 1999 to 2023 estimated from the GLMM for the
%     data including 2022 (top) and excluding 2022 (bottom).}
%   \label{fig:VenueEffects}
% \end{figure}

% The uniqueness of 2022 is further demonstrated by Figure~\ref{fig:VenueEffects},
% which shows the values of the venue random effects.  The temporal patterns
% of the two sets of effects look very similar, except that the positive effects
% estimated with 2022 included have higher magnitude than those estimated without
% 2022, which compensates the negative effect of the 2022 with a large magnitude.
% Both plots agree to some extent with the pattern of the boxplots in
% Figure~\ref{fig:Boxplot}. The mismatch is due to the heat effect, since the venue
% effect only captures a smaller part of the variation than the heat effect as
% evident from their standard deviations. It is possible to calculate the extremity
% of the 2022 heat effect as we know that the mean of the heat effects is $0$,
% the standard deviation is $0.42$ as given by Table~\ref{tab:Gamma_parameters},
% and the estimate of the 2022 heat effect is $-0.93$. We find the tail probability
% to be $0.0128$ indicating there is a low chance of observing a venue effect as
% extreme as 2022.


\jy{Could you also test sensitivity for excluding the positive but
  fault RTs?}
\of{Yes I explained the code is at the bottom of gg.R and ran code without 2022: for 0.1 the
probability is $1.97\cdot10^{-3}$ versus $2.76\cdot10^{-3}$. For 0.09: the
probability is $3.53\cdot10^{-4}$ versus $4.95\cdot10^{-4}$. For 0.08: the
probability is $4.93\cdot10^{-5}$ versus $6.84\cdot10^{-5}$. I assume we want
to report these in the supplemental?}

\begin{table}
  \centering
  \caption{Probabilities of observing RTs less than threshold 0.08,
  0.09, and 0.10 seconds based on the two effect model.}
  \begin{tabular}{c c c c}
   \toprule
   Data Set & Threshold 0.08 & Threshold 0.09 & Threshold 0.10  \\
   \midrule
   Excluding 2022 & $5.31\cdot10^{-5}$ & $3.53\cdot10^{-4}$ &  $1.94\cdot10^{-3}$  \\
   Including 2022 & $6.84\cdot10^{-5}$ & $4.95\cdot10^{-4}$ & $2.76\cdot10^{-3}$ \\
   \bottomrule
  \end{tabular}
  \label{tab:Sim_probability}
\end{table}


The fitted GG GAMLSS model with both venue- and heat-level
random effects provides a framework for assessing how extreme RTs
below certain thresholds are. The probability of observing a RT
below a given threshold, assuming no intentional false starts, was approximated
by generating 10 million realizations from the fitted model.
Table~\ref{tab:Sim_probability} summarizes the probabilities of observing
RTs below 0.08, 0.09, and 0.10 seconds under two scenarios: one
excluding and the other including data from 2022. Excluding 2022 slightly
reduces the probability of observing a fast RT, but the difference
is small. For example, the probability of a RT below 0.10 seconds
decreases from $2.76 \cdot 10^{-3}$ (approximately one in 362 starts) to
$1.94 \cdot 10^{-3}$ (approximately one in 515 starts) when 2022 is excluded.
Lowering the RT threshold from 0.10 to 0.08 seconds drastically
reduces the likelihood of observing a RT below the barrier, with
the probability dropping from one in every 362 starts (at 0.10 seconds) to one
in every 14620 starts (at 0.09 seconds) and one in every 146198 starts (at 0.08
seconds) when 2022 is included. These results highlight the rarity of extremely
fast RTs and substantiate the recommendations of \citet{komi2009iaaf}
to carefully consider the selection of RT thresholds.


\begin{table}
  \centering
  \caption{Suggested RT barriers based on tail probabilities.}
  \begin{tabular}{c c c}
   \toprule
   Data Set & Tail probability  $10^{-3}$ & Tail probability $10^{-4}$ \\
   \midrule
   Excluding 2022 & $0.096$ & $0.083$ \\
   Including 2022 & $0.094$ & $0.082$ \\
   \bottomrule
  \end{tabular}
  \label{tab:Sim_time}
\end{table}

Utilizing the same model, we can determine suitable RT barriers
based on the probability of observing a time below the barrier. As shown in
Table~\ref{tab:Sim_time}, including the 2022 data suggests a RT
barrier of 0.094 seconds to maintain a 0.1\% chance of observing an
exceptionally fast RT, while a stricter threshold of 0.082 seconds
is needed to limit this probability to 0.01\%. Excluding the 2022 data results
in slightly higher thresholds of 0.096 and 0.083 seconds for the respective
probability levels. These results indicate that while the inclusion of 2022
data slightly reduces the recommended barrier, the magnitude of the difference
is relatively small. This approach allows for tailoring RT
thresholds to desired levels of false positive rates, balancing fairness and
precision in disqualification criteria.


\section{Discussion}\label{sec:concludingremarks}


This study sought to answer two primary questions: were athletes who
competed at the 2022 World Track and Field Championships faster than
at other races, and is the 0.1-second RT barrier fair?
Regarding the first question, our analyses suggest that athletes were
indeed faster at the 2022 World Championships compared to comparable
meets. Ideally, a centralized database containing RTs from
all World Athletics-certified meets would have allowed for a more
comprehensive analysis. Currently, however, RTs are seldom
included in the results available on the World Athletics website. This
limitation constrained our ability to evaluate athletes like Devon
Allen, as data from his races leading up to the World Championships in
2022 were not accessible. A more robust dataset would enable deeper
insights by allowing comparisons across a larger set of races.


In addressing the second question, our analyses of the GAMLSS model
reveal that the 0.1-second RT barrier, while rare, may not
be as extraordinary  as traditionally assumed. For men,
Table~\ref{tab:Sim_time} shows that reaction  times below 0.1 seconds
occur with a probability of approximately one in 362 starts  when
including the 2022 data. Lowering the threshold to 0.08 seconds
drastically  reduces this likelihood, supporting the idea that the
current barrier could be adjusted to reflect more realistic
probabilities of false starts. A similar pattern is observed for
women, as detailed in the Supplementary Material, where reaction
times below 0.1 seconds are exceedingly rare for the 100-meter dash
and 100-meter  hurdles. However, the uniformity of the 0.1-second
barrier for both men and women  is questionable, given numerous
studies documenting gender differences in reaction  times
\citep[e.g.,][]{lipps2011implications, babicc2009reaction,
  panoutsakopoulos2020gender}. These studies suggest that the current
threshold may  unfairly penalize men, for whom sub-0.1-second reaction
times are more probable. \citet{brosnan2017effects} advocate for
gender-specific barriers, a position that  aligns with our findings
and highlights the importance of tailoring thresholds to  biological
distinctions.


This study provides a statistical framework to examine Devon Allen’s
disqualification at the 2022 World Track and Field Championships,
offering insights rather than drawing definitive conclusions about
potential equipment malfunction. Our findings indicate that reaction
times at the 2022 Championships were, on average, faster than at other
competitions, as evidenced by the significant p-values in
Table~\ref{tab:Clusrankresults}. Additionally, the GAMLSS results
suggest that the 0.1-second barrier may not be as stringent as
previously believed. Based on Table~\ref{tab:Sim_time}, a stricter
threshold of 0.08 seconds could be considered, allowing athletes like
Allen to react swiftly without undue risk of disqualification. While
this analysis provides a rigorous statistical perspective, it does not
consider biomechanical factors, such as individual variability in
neuromuscular response times or the role of starting block sensors in
detecting pressure changes, which may offer more direct evidence of
reaction capabilities. In summary, while the results designate 2022 as
an anomalous year, Allen’s time, despite resulting in
disqualification, may not be categorically extreme.


\section*{Supplementary Material}
The data and R code used for the analysis are available in a compressed file for
ease of reproducibility.

\bibliographystyle{chicago}
\bibliography{citations}


\end{document}
