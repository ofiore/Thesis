%\documentclass[12pt, letterpaper, titlepage]{article}
\documentclass[12pt, letterpaper]{article}

\usepackage{amsmath, amsfonts}
\usepackage{booktabs}
\usepackage{amsthm}
\usepackage{graphicx}
\usepackage[margin=1in]{geometry}
\usepackage{hyperref}
\usepackage{cleveref}
\hypersetup{colorlinks = true, linkcolor = blue, citecolor=blue, urlcolor = blue}
\usepackage{natbib}
\usepackage{float}
\usepackage{setspace}
\usepackage{pdfpages}
\usepackage{lineno}
\usepackage{mwe}
\usepackage{comment}
\linenumbers*[1]
% %% patches to make lineno work better with amsmath
\newcommand*\patchAmsMathEnvironmentForLineno[1]{%
 \expandafter\let\csname old#1\expandafter\endcsname\csname #1\endcsname
 \expandafter\let\csname oldend#1\expandafter\endcsname\csname end#1\endcsname
 \renewenvironment{#1}%
 {\linenomath\csname old#1\endcsname}%
 {\csname oldend#1\endcsname\endlinenomath}}%
\newcommand*\patchBothAmsMathEnvironmentsForLineno[1]{%
 \patchAmsMathEnvironmentForLineno{#1}%
 \patchAmsMathEnvironmentForLineno{#1*}}%

\AtBeginDocument{%
 \patchBothAmsMathEnvironmentsForLineno{equation}%
 \patchBothAmsMathEnvironmentsForLineno{align}%
 \patchBothAmsMathEnvironmentsForLineno{flalign}%
 \patchBothAmsMathEnvironmentsForLineno{alignat}%
 \patchBothAmsMathEnvironmentsForLineno{gather}%
 \patchBothAmsMathEnvironmentsForLineno{multline}%
}

% control floats
\renewcommand\floatpagefraction{.9}
\renewcommand\topfraction{.9}
\renewcommand\bottomfraction{.9}
\renewcommand\textfraction{.1}
\setcounter{totalnumber}{50}
\setcounter{topnumber}{50}
\setcounter{bottomnumber}{50}

\newcommand{\jy}[1]{\textcolor{blue}{JY: #1}}
\newcommand{\eds}[1]{\textcolor{red}{EDS: (#1)}}
\newcommand{\of}[1]{\textcolor{violet}{OF: #1}}

% NOTE: To produce blinded version, replace "0" with "1" below.
\newcommand{\blind}{0}

%\title{On Devon Allen's Disqualification at the 2022 World Track and Field
%Championships}
%
%\author{Owen Fiore\\
%%   \href{mailto:owen.fiore@uconn.edu}
%% {\nolinkurl{owen.fiore@uconn.edu}}\\
  %Elizabeth D. Schifano\\
  %Jun Yan\\[1ex]
  %Department of Statistics, University of Connecticut\\
%}
%\date{}

\begin{document}

\title{\bf Supplement to ``On Devon Allen's Disqualification at the 2022 World Track and Field Championships''}

\if0\blind
{
  \author{Owen Fiore, %\\
%   \href{mailto:owen.fiore@uconn.edu}
% {\nolinkurl{owen.fiore@uconn.edu}}\\
  Elizabeth D. Schifano, %\\
  Jun Yan\\[1ex]
  Department of Statistics, University of Connecticut\\
}
} \fi

\if1\blind
{
  \bigskip
  \bigskip
  \bigskip
  \author{Anonymous Authors}
  \bigskip
} \fi

\maketitle 

\section{Rank-Based Comparison with Pooled Data}

As an alternative to the methods described in Section 3.1 of the main paper, it
is possible to combine all men's and women's reaction time (RT) data for each of 
the three competition
comparisons.  Thus we shrink our analyses from six to three, but each analysis
is roughly twice as big as previously.
For the 2022 national versus
international comparison, there were 160 RTs from 35
athletes and the asymptotic test result was a p-value of $1.94 \cdot 10^{-7}$.
For the 2019 versus 2022 comparison,
there were 258 RTs from 65 athletes and the asymptotic
test result was a p-value of $3.56 \cdot 10^{-8}$. 
For the 2023 versus 2022 comparison, there were 343 RTs
from 92 athletes and the asymptotic test result was a p-value of
$4.99 \cdot 10^{-12}$.  As a result of the larger sample sizes, the 
permutation-based tests were very computationally expensive to run and so we 
used a smaller number of permutations ($100,000$ instead of $1,000,000$).  All
three permutation tests had a p-value of $1 \cdot 10^{-5}$, which is the smallest
possible value given $100,000$ permutations.  When taken together
with the asymptotic results, the message is very clear. These are all
highly significant test results that show substantial differences in average RT
for athletes competing at multiple championship-level competitions.


\section{GAMLSS Results for Women Data}

We also apply the RT barrier analysis described in Section 3.2 to women's data,
fitting the same model to women's RT data from 2001 to 2023.
The RT data for women is visualized in Figure~\ref{fig:WomensBoxplot}.
Similar to the men's data, RTs from 2022 appear lower than in other
years. After removing one obvious outlier, which was a disqualified
reaction time, the same model for men's data fits the women's data
reasonably well.

\begin{figure}[tbp]
  \centering
  \includegraphics[width=\textwidth]{WomensBoxplot}
  \caption{The RTs from 2021 to 2023 for the women's 100 meter hurdle
  and 100 meter dash.}
  \label{fig:WomensBoxplot}
\end{figure}


\begin{table}
  \centering
  \caption{Estimated fixed-effect parameters with standard errors in
    parentheses and estimated standard deviations of the random effects from the men's and
    women's fitted GG distribution with venue level random
    effects in $\mu$ and heat level random effects in $\sigma$. $n$ denotes
    size of the data.}
  \label{tab:womensfit}
  \begin{tabular}{c c c c c c c}
    \toprule
    Dataset & $n$ & $\beta_0$ & $\gamma_0$ & $\nu$ & $\tau_v$ & $\tau_h$ \\
    \midrule
    Women's & 732 & $-$1.921 (0.007) & $-$2.071 (0.028) & $-$3.691 (0.665) & 0.057 & 0.111 \\
    Men's & 776 & $-$1.910 (0.005) & $-$2.200 (0.027) & $-$1.178 (0.447) & 0.058 & 0.320 \\
    \bottomrule
  \end{tabular}
\end{table}

The fitted parameters in comparison with those from men's data are
summarized in Table~\ref{tab:womensfit}.  One notable aspect is that while
the venue effect standard deviation is nearly identical, the women's heat effect
standard deviation is smaller and $\nu$ is much larger.
This suggests that men’s races exhibit greater 
variability in RTs, possibly due to faster athletes influencing others 
to react more quickly in certain instances. The consistency in the venue effect 
standard deviation across men’s and women’s data indicates that the venue effect 
is not only statistically significant but also consistent in magnitude across 
genders. These findings highlight the potential impact of competition dynamics 
on heat variability and the robustness of venue-level effects.


\begin{table}
  \centering
  \caption{Probabilities of observing RTs less than threshold 0.08,
  0.09, and 0.10 seconds based on the men's and women's
    fitted GG GAMLSS model with both venue- and heat-level
random effects.}
  \begin{tabular}{c c c c}
   \toprule
   Data Set & Threshold 0.08 & Threshold 0.09 & Threshold 0.10  \\
   \midrule
   Women's & $1\cdot10^{-7}$ & $1.12\cdot10^{-5}$ &  $5.46\cdot10^{-4}$  \\
   Men's   & $6.84\cdot10^{-5}$ & $4.95\cdot10^{-4}$ & $2.76\cdot10^{-3}$ \\
   \bottomrule
  \end{tabular}
  \label{tab:Sim_prob_women}
\end{table}

We also repeat the simulation methods described in the paper to evaluate
the probability of an extreme RT for women.  Table~\ref{tab:Sim_prob_women}
compares the men's and women's results of observing RTs less than
0.08, 0.09, and 0.1 seconds.  We find across all three thresholds that
women have a lower probability of having a fast RT. The results are slightly
different from those from the men's data, which echoes existing studies
reporting gender differences in RTs \citep[e.g.,][]{lipps2011implications,
babicc2009reaction, panoutsakopoulos2020gender}.



\begin{table}
  \centering
  \caption{Suggested RT barriers based on tail probabilities.}
  \begin{tabular}{c c c c}
   \toprule
   Data Set & Tail probability  $10^{-2}$ & Tail probability  $10^{-3}$ & Tail probability $10^{-4}$ \\
   \midrule
   Women's & $0.111$ & $0.102$ & $0.095$ \\
   Men's   & $0.108$ & $0.094$ & $0.082$ \\
   \bottomrule
  \end{tabular}
  \label{tab:Sim_time_women}
\end{table}

To evaluate a fair RT barrier for women, we compared suggested 
barriers for men and women based on tail probabilities, as shown in 
Table~\ref{tab:Sim_time_women}. The results indicate that men are more likely 
to be disqualified under the current uniform 0.1-second threshold due to their 
generally faster RTs. This suggests that the same RT standard may not 
have equivalent implications for men and women. Potential adjustments could 
involve raising the barrier for women to align with men’s disqualification rates 
or lowering the barrier for men to match women’s rates. However, further research 
is needed to validate these findings and explore their broader implications. These 
results contribute to ongoing discussions about RT thresholds and emphasize the 
importance of statistical evidence in guiding decisions about competition fairness.


\section{Results from Data Excluding Positive Disqualified RTs}

An earlier iteration of the paper fit a model that did not include RTs from
athletes who were disqualified or did not finish but still registered a RT.  
However, it was ultimately decided to include these times to better estimate the 
left tail of the distribution and more accurately predict the probability of a 
low RT, as described in the main paper. 
We did not include negative RTs, however, as these represent a mistake of the
runner for starting before the gun is fired and are thus meaningless in our
objective to determine a fair RT barrier.  Not all of those disqualified were
disqualified because of breaking the 0.1 reaction time barrier; there are many
reasons why an athlete may be disqualified, with the most notable being failed
drug tests and lane violations.  

Nonetheless, in this section,
we exclude all disqualified RTs to see their effect on the probability of an
extreme RT. We otherwise fit an identical generalized Gamma model to the men's 
dash and hurdles RT data (including 2022), as presented in Section 3.2,
to determine how sensitive our model is to the inclusion/exclusion of these times.


\begin{table}
  \centering
   \caption{Probabilities of observing RTs less than threshold 0.08,
   0.09, and 0.10 seconds based on the
     fitted GG GAMLSS model with both venue- and heat-level
 random effects.}
   \begin{tabular}{c c c c}
    \toprule
    Data Set & Threshold 0.08 & Threshold 0.09 & Threshold 0.10  \\
    \midrule
    Without DQs & $4.93\cdot10^{-5}$ & $3.53\cdot10^{-4}$ &  $1.97\cdot10^{-3}$  \\
    With DQs & $6.84\cdot10^{-5}$ & $4.95\cdot10^{-4}$ & $2.76\cdot10^{-3}$ \\
    \bottomrule
   \end{tabular}
   \label{tab:DQSim_probability}
 \end{table}


Table~\ref{tab:DQSim_probability} shows the effect of removing disqualified (DQ)
times from the analysis.  The probability of observing extreme RTs is lower when
we remove the 17 observations.  While the probability of observing an extreme
RT is less when we remove the RTs of disqualified athletes, many of the 
conclusions remain the same: the current standards for disqualification are not
grounded in statistical analysis, and there appear to be unequal standards for men
and women.

\bibliographystyle{apalike}
\bibliography{citations}


\end{document}
