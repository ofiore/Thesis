%\documentclass[12pt, letterpaper, titlepage]{article}
\documentclass[12pt, letterpaper]{article}

\usepackage{amsmath, amsfonts}
\usepackage{booktabs}
\usepackage{amsthm}
\usepackage{graphicx}
\usepackage[margin=1in]{geometry}
\usepackage{hyperref}
\usepackage{cleveref}
\hypersetup{colorlinks = true, linkcolor = blue, citecolor=blue, urlcolor = blue}
\usepackage{natbib}
\usepackage{float}
\usepackage{setspace}
\usepackage{pdfpages}
\usepackage{lineno}
\usepackage{mwe}
\usepackage{comment}
\linenumbers*[1]
% %% patches to make lineno work better with amsmath
\newcommand*\patchAmsMathEnvironmentForLineno[1]{%
 \expandafter\let\csname old#1\expandafter\endcsname\csname #1\endcsname
 \expandafter\let\csname oldend#1\expandafter\endcsname\csname end#1\endcsname
 \renewenvironment{#1}%
 {\linenomath\csname old#1\endcsname}%
 {\csname oldend#1\endcsname\endlinenomath}}%
\newcommand*\patchBothAmsMathEnvironmentsForLineno[1]{%
 \patchAmsMathEnvironmentForLineno{#1}%
 \patchAmsMathEnvironmentForLineno{#1*}}%

\AtBeginDocument{%
 \patchBothAmsMathEnvironmentsForLineno{equation}%
 \patchBothAmsMathEnvironmentsForLineno{align}%
 \patchBothAmsMathEnvironmentsForLineno{flalign}%
 \patchBothAmsMathEnvironmentsForLineno{alignat}%
 \patchBothAmsMathEnvironmentsForLineno{gather}%
 \patchBothAmsMathEnvironmentsForLineno{multline}%
}

% control floats
\renewcommand\floatpagefraction{.9}
\renewcommand\topfraction{.9}
\renewcommand\bottomfraction{.9}
\renewcommand\textfraction{.1}
\setcounter{totalnumber}{50}
\setcounter{topnumber}{50}
\setcounter{bottomnumber}{50}

\newcommand{\jy}[1]{\textcolor{blue}{JY: #1}}
\newcommand{\eds}[1]{\textcolor{red}{EDS: (#1)}}
\newcommand{\of}[1]{\textcolor{violet}{OF: #1}}

% NOTE: To produce blinded version, replace "0" with "1" below.
\newcommand{\blind}{0}

%\title{On Devon Allen's Disqualification at the 2022 World Track and Field
%Championships}
%
%\author{Owen Fiore\\
%%   \href{mailto:owen.fiore@uconn.edu}
%% {\nolinkurl{owen.fiore@uconn.edu}}\\
  %Elizabeth D. Schifano\\
  %Jun Yan\\[1ex]
  %Department of Statistics, University of Connecticut\\
%}
%\date{}

\begin{document}

\title{\bf Supplement to ``On Devon Allen's Disqualification at the 2022 World Track and Field Championships''}

\if0\blind
{
  \author{Owen Fiore, %\\
%   \href{mailto:owen.fiore@uconn.edu}
% {\nolinkurl{owen.fiore@uconn.edu}}\\
  Elizabeth D. Schifano, %\\
  Jun Yan\\[1ex]
  Department of Statistics, University of Connecticut\\
}
} \fi

\if1\blind
{
  \bigskip
  \bigskip
  \bigskip
  \author{Anonymous Authors}
  \bigskip
} \fi

\maketitle 

\section{Rank-Based Comparison with Pooled Data}

\jy{Keep the order consistent with the text; do the same in the code.}

As an alternative to the methods described in Section 3.1 of the main paper, it
is possible to combine all men's and women's reaction time (RT) data for each of 
the three competition
comparisons.  Thus we shrink our analyses from six to three, but each analysis
is roughly twice as big as previously.  For the 2019 versus 2022 comparison,
there were 258 RTs from 65 athletes and the asymptotic
test result was a p-value of $3.56 \cdot 10^{-8}$. For the 2022 national versus
international comparison, there were 160 RTs from 35
athletes and the asymptotic test result was a p-value of $1.94 \cdot 10^{-7}$.
For the 2023 versus 2022 comparison, there were 343 RTs
from 92 athletes and the asymptotic test result was a p-value of
$4.99 \cdot 10^{-12}$.  As a result of the larger sample sizes, the code was
very computationally expensive to run and thus we ran it on a smaller size.  All
three permutation tests had a p-value of $1 \cdot 10^{-5}$, which is the smallest
possible value given the permutation test size of $100,000$.  When taken together
with the asymptotic results, the results becoems very clear. These are all
highly significant test results that show substantial differences in average RT
for athletes competing at multiple championship-level events.


\section{GAMLSS Results for Women Data}

We can repeat our reaction time barrier analysis described in Section 3.2, but
fit the same model to women's RT data from 2001 to 2023.  Once again we utilize
the power of the venue effect within $\mu$ and a heat effect within $\sigma$.
\eds{what is meant by `power' here?} 
The RT data for women is visualized in Figure~\ref{fig:WomensBoxplot}.
Similar to the men's data, RTs from 2022 appear lower than in other
years.



\begin{figure}[tbp]
  \centering
  \includegraphics[width=\textwidth]{WomensBoxplot}
  \caption{The RTs from 2021 to 2023 for the women's 100 meter hurdle
  and 100 meter dash.}
  \label{fig:WomensBoxplot}
\end{figure}

We examined several models in the generalized Gamma model and found that the
best model for women involved a venue effect in $\mu$ and heat effect in
$\sigma$.  Thus the best model for men was also the best model for women.

\begin{table}
  \centering
  \caption{Estimated fixed-effect parameters with standard errors in
    parentheses and estimated variance of the random effects from the men's and
    women's fitted GG distribution with venue level random
    effects in $\mu$ and heat level random effects in $\sigma$. $n$ denotes
    size of the data.}
  \label{tab:womensfit}
  \begin{tabular}{c c c c c c c}
    \toprule
    Dataset & $n$ & $\beta_0$ & $\gamma_0$ & $\nu$ & $\tau_v$ & $\tau_h$ \\
    \midrule
    Women's & 732 & $-$1.921 (0.007) & $-$2.071 (0.028) & $-$3.691 (0.665) & 0.057 & 0.111 \\
    Men's & 776 & $-$1.910 (0.005) & $-$2.200 (0.027) & $-$1.178 (0.447) & 0.058 & 0.320 \\
    \bottomrule
  \end{tabular}
\end{table}

One of the most interesting aspects of Table~\ref{tab:womensfit} is that while
the venue effect standard deviation is nearly identical, the women's heat effect
standard deviation is smaller and $\nu$ is much larger.  If men's races have
more variability in times, it seems possible that in men's races that faster
athletes can influence other athletes to react faster in certain instances.
It is also interesting that the standard deviation of the venue effect is
nearly identical for both men's and women's data, indicating that the venue
effect is not only statistically significant but is consistent in strength
across men and women.

We can repeat the simulation methods described in the main paper to evaluate
the probability of an extreme reaction time.  Table~\ref{tab:Sim_prob_women}
compares the men's and women's results of observing reaction times less than
0.08, 0.09, and 0.1 seconds.  We find that across all three comparisons that
women have a lower probability of having a fast time. The results are slightly
different from those from the men's data, which echoes existing studies
reporting gender differences in RTs \citep[e.g.,][]{lipps2011implications,
babicc2009reaction, panoutsakopoulos2020gender}.

\begin{table}
  \centering
  \caption{Probabilities of observing RTs less than threshold 0.08,
  0.09, and 0.10 seconds based on the men's and women's
    fitted GG GAMLSS model with both venue- and heat-level
random effects.}
  \begin{tabular}{c c c c}
   \toprule
   Data Set & Threshold 0.08 & Threshold 0.09 & Threshold 0.10  \\
   \midrule
   Women's & $1\cdot10^{-7}$ & $1.12\cdot10^{-5}$ &  $5.46\cdot10^{-4}$  \\
   Men's   & $6.84\cdot10^{-5}$ & $4.95\cdot10^{-4}$ & $2.76\cdot10^{-3}$ \\
   \bottomrule
  \end{tabular}
  \label{tab:Sim_prob_women}
\end{table}


Now we need to perform alter the interpretation of the analyis to determine
what a fair reaction time barrier for women should be.
Table~\ref{tab:Sim_time_women} shows the difference in men's and women's
suggested reaction time barrier.  What we see is that by World Athletics setting
the same standard for men and women, their rules are not fair, in that men are
more likely to be disqualified because they are more likely to react quickly.
One of the biggest takeaways of the reaction time barrier analysis is that the
current standards imposed by World Athletics are not grounded in statistical
analysis and that they should be.  It seems reasonable to either raise the
women's barrier so that they are disqualified at a similar rate as men or to
lower the men's barrier so they are disqualified at a similar rate as women.  But
having equal but not neccessarily fair standards does not seem reasonable when
this study's results echoed results of numerous other papers asking World
Athletics to change the standards.

\begin{table}
  \centering
  \caption{Suggested RT barriers based on tail probabilities.}
  \begin{tabular}{c c c c}
   \toprule
   Data Set & Tail probability  $10^{-2}$ & Tail probability  $10^{-3}$ & Tail probability $10^{-4}$ \\
   \midrule
   Women's & $0.111$ & $0.102$ & $0.095$ \\
   Men's   & $0.108$ & $0.094$ & $0.082$ \\
   \bottomrule
  \end{tabular}
  \label{tab:Sim_time_women}
\end{table}




\section{Results from Data Excluding Positive Disqualified RTs}

An earlier iteration of the paper fit a model that did not include RTs from
athletes who were disqualified or did not finish but still registered a RT.  
However, it was ultimately decided to include these times to better estimate the 
left tail of the distribution and more accurately predict the probability of a 
low RT, as described in the main paper.  
We did not include negative RTs, however, as these represent a mistake 
of the runner for starting before
the gun is fired and are thus meaningless in our objective to
determine a fair RT barrier.  Not all of those disqualified were
disqualified because of breaking the 0.1 reaction time barrier; there are many
reasons why an athlete may be disqualified, with the most notable being failed
drug tests and lane violations.  Nonetheless, in this section,
we exclude all disqualified RTs to see their effect on the probability of an
extreme RT.  We fit an identical model to the generalized Gamma model
presented in Section 3.2 to examine differences results.

\begin{table}
  \centering
  \caption{Probabilities of observing RTs less than threshold 0.08,
  0.09, and 0.10 seconds based on the
    fitted GG GAMLSS model with both venue- and heat-level
random effects.}
  \begin{tabular}{c c c c}
   \toprule
   Data Set & Threshold 0.08 & Threshold 0.09 & Threshold 0.10  \\
   \midrule
   Without DQs & $4.93\cdot10^{-5}$ & $3.53\cdot10^{-4}$ &  $1.97\cdot10^{-3}$  \\
   With DQs & $6.84\cdot10^{-5}$ & $4.95\cdot10^{-4}$ & $2.76\cdot10^{-3}$ \\
   \bottomrule
  \end{tabular}
  \label{tab:DQSim_probability}
\end{table}

\jy{Men's data or pooled data?}
\eds{with or without 2022?}
\of{I dont know if we have time for this or if this fits the scope of what we
are working on.}

Table~\ref{tab:DQSim_probability} shows the effect of removing disqualificated
times from the analysis.  The probability of observing extreme RTs is lower when
we remove the 17 observations.

\bibliographystyle{apalike}
\bibliography{citations}


\end{document}
