\documentclass[12pt]{article}
\usepackage[margin=1in]{geometry}
\usepackage{natbib}
\usepackage{hyperref}

\usepackage{listings}
\usepackage{rotating, graphicx}
\graphicspath{{./}, {./image/}}
\usepackage{booktabs, natbib}
% \usepackage{breakurl}
% \usepackage [english]{babel}
\usepackage{amsmath, amsbsy, amsthm, epsfig, epsf, psfrag, graphicx,
amssymb, enumerate}
\usepackage{bm}
\usepackage{multirow, multicol}

\usepackage[dvipsnames]{xcolor}
\definecolor{darkblue}{rgb}{0.1, 0.2, 0.6}


\newcommand{\jy}[1]{\textcolor{orange}{JY: #1}}
\newcommand{\eds}[1]{\textcolor{blue}{(EDS: #1)}}
\newcommand{\of}[1]{\textcolor{purple}{(OF: #1)}}

\sloppy

% \usepackage{csquotes}
% \usepackage [autostyle, english = american]{csquotes}
% \MakeOuterQuote{"}

% \usepackage{bibentry}
\newenvironment{comment}%
{\begin{quotation}\noindent\small\it\color{darkblue}\ignorespaces%
}{\end{quotation}}


\begin{document}

\begin{center}
  {\Large\bf TAS-23-296.R1: Response to the Comments}
\end{center}

We sincerely thank the Editor and Associate Editor for tentatively accepting our
submission and are proud to present our most recent revisions.


The manuscript has been revised accordingly with the following most
notable changes:
\begin{enumerate}
\item
  The manuscript has been re-structured to have 4 sections: introduction,
  analysis of 2022 times, analysis of a proper reaction barrier, and discussion.
  Within the second and third sections there are data, methods, and results
  subsections.
\item
  General revisions of tidying the manuscript.

Point-by-point responses to the comments are as follows, with the
comments quoted in \emph{\color{darkblue} italic and blue}.
\end{enumerate}


\subsection*{To Reviewer 1}

\begin{comment}
This revision is a solid improvement on the original paper exploring the
anomalous reaction times at the 2022 World Championships and the implications of
using 0.100 seconds as the cut off for acceptable reaction times. The updated
paper addresses many of my original points of feedback, including, notably, data
from more events, improving the statistical power and generalizability of the
results. I also think the expansion of the features considered in the model is
beneficial (although, I do think considering a Bayesian approach to the
distribution estimation could be beneficial).

However, I still believe there is room to improve the overall quality of the
paper, both from a technical/methodological perspective and from a more general
clarity and organization perspective.


\end{comment}

We thank the reviewer for the accurate summary of our contributions and
understand there are some smaller issues to still address.



\begin{comment}
First, I think the description of the data would benefit from an easier to
follow organization system. Currently, headings repeat and the overall
difference between the usage of data from the World Championships in one setting
instead of the other is not as easily understood as it could or probably should
be. I would recommend a more intuitive organization of this material and, if
possible, the construction and inclusion of a table summarizing the various data
details in each analysis. As a reader, I would like a fast way to keep track of
the important details in each case.
\end{comment}

We understand there were issues about distinguishing between the datasets being
used for each section.  After re-grouping the paper, each problem's data, 
methods, and results section are consecutive, leading to increased reader
comprehension.  The dataset describing the 2022 analysis is found in
the last paragraph of page 5 and the dataset describing the RT barrier analysis
is found in the first paragraph of section 3.1


\begin{comment}
Second, I think the results themselves would benefit from organizing around the
question or problem being addressed instead of the method. Organizing by method
makes it harder for your reader to connect the results to the research questions
or prompts.
\end{comment}

Thank you for this suggestion, we have decided to re-organize our paper as a
result. Now, the first half of the paper is dedicated to the analysis of
athletes competing faster in 2022 and the second half is dedicated to the
analysis of the reaction time barrier, with an introduction and discussion to
show the connectivity of these problems.



\begin{comment}
Third, I think the manuscript as a whole needs some general editing and cleaning
up before publication. For example, there are discrepancies in the language used
in some of the tables - in Table 1 the groups are labeled both "Men/Women" and
"Men/Female" when consistency would be helpful for the readers. Similarly, Table
2 uses the model notation to as column headers, making it hard to quickly assess
what it is communicating. A stronger organizational approach could also help the
overall clarity of the article.
\end{comment}

The manuscript has been revised again and the langugage in Table 1 is now
consistent.  The Table 2 nomenclature is now easier to interpret because the
methods section describing the symbols is now placed on the page before. 



\begin{comment}
Finally, I think it is important that the article consider the possible causes
of differences between the competitions that are not accounted for in the
matched-pairs design. As I understand the approach, the analysis does not
account for two important potential sources of change in the reaction times::
seasonality and age. Seasonality could come into play if athletes' reaction
times improve as they attempt to peak in their season (likely at the highest
level competition they will participate in). If athletes expect to compete at
the World Championships, they might not be as fit or prepared at their national
championships and produce slower reaction times. Similarly, age could lead to
differences in an athletes' reaction times as older athletes will react slower
than younger athletes. While it would be great to see these features accounted
for directly, I think at a minimum they and their likely influences on the
results, should be discussed in the overall limitations of the analysis.
\end{comment}


We have added additional explanations in the discussion section to address both
seasonality and age, although we don't think either one will be largely
impactful.  By including times from athletes in both 2019 and 2022, there are
athletes who got older from 2019 to 2022 but athletes in 2022 will be younger
than they were in 2023.  Including both comparisons is meaningful as it helps
diminish this effect.  Likewise while seasonality may impact the 2022 national
level comparison, the 2019 and 2023 comparisons mitigate this factor.

\begin{comment}
Altogether, I think the technical work here is solid and shows that the reaction
times at the 2022 World Championships were identifiably faster than at other
major competitions and that 0.100 threshold for acceptable reaction times is
likely too conservative of a standard given athletes' natural abilities. The
largest areas of need are mainly in the organization and overall clarity of the
paper.
\end{comment}

Thank you for your comments.


\bibliographystyle{apalike}
\bibliography{citations}
\end{document}
