\documentclass[12pt]{article}
\usepackage[margin=1in]{geometry}
\usepackage{natbib}
\usepackage{hyperref}

\usepackage{listings}
\usepackage{rotating, graphicx}
\graphicspath{{./}, {./image/}}
\usepackage{booktabs, natbib}
% \usepackage{breakurl}
% \usepackage [english]{babel}
\usepackage{amsmath, amsbsy, amsthm, epsfig, epsf, psfrag, graphicx, 
amssymb, enumerate}
\usepackage{bm}
\usepackage{multirow, multicol}

\usepackage[dvipsnames]{color}
\definecolor{darkblue}{rgb}{0.1, 0.2, 0.6}

\newcommand{\jy}[1]{\textcolor{red}{JY: #1}}
\newcommand{\eds}[1]{\textcolor{blue}{(EDS: #1)}}
\newcommand{\mc}[1]{\textcolor{green}{(MC: #1)}}

\sloppy

% \usepackage{csquotes}
% \usepackage [autostyle, english = american]{csquotes}
% \MakeOuterQuote{"}

% \usepackage{bibentry}
\newenvironment{comment}%
{\begin{quotation}\noindent\small\it\color{darkblue}\ignorespaces%
}{\end{quotation}}


\begin{document}

\begin{center}
  {\Large\bf Response to the Comments}
\end{center}

We extend our gratitude to the Editor and Associate Editor for 
granting us the opportunity to revise this manuscript. We also want to
express our appreciation to the reviewer for their valuable comments. 


The manuscript has been revised accordingly with the most notable change being 
the addition of Section 5 (Classroom Implementation). We believe this revision 
has enhanced the both the quality and utility of this paper.


Point-by-point responses to the comments are as follows, with the
comments quoted in \emph{\color{darkblue} italic and blue}.

\subsection*{To Associate Editor}

\begin{comment}
Using Devon Allen's disqualification at the 2022 World Track and Field
championships as a hook, this paper poses two questions. First, were reaction
times at the 2022 World Track and Field Championships faster than they had been
earlier in 2022 national-level competitions, the 2019 World Track and Field
Championships, and the 2023 World Track and Field Championships? Second, based
on modeled reaction times, is the current standard of a 0.1 second threshold for
reaction times before disqualification too conservative?
\end{comment}


Thank you very much for your encouraging comments and great suggestion.  
We have now included a new section, \textit{Section 5 Classroom Implementation},
that discusses how elements of this manuscript have been incorporated into both 
graduate- and undergraduate-level teaching at the authors' home university. 
 

For blinding purposes, course numbers, author initials and websites, and 
university name have been redacted, but all this information appears in the 
unblinded version.  In brief, this material has been covered at the graduate
level in a statistical computing course, open to all graduate students in 
Statistics and others with permission, that covers the bootstrap. Detailed R 
code and graphical illustrations were provided to students in the open-access 
course notes to elucidate the distinctions between employing null
hypotheses with and without defined parameters.  This link is included in the 
unblinded version of the paper; for the blinded review, we printed the 
relevant portion of the notes to pdf and included this blinded pdf file in the 
Supplemental Materials. %\eds{Not sure how we can blind your github 
%website for the submission unless we actually don't include it in the blinded 
%version. Could we print to PDF the relevant section of the notes and include as 
%a supplement for review purposes?} 


Similarly at the undergraduate-level, students in an R-based applied regression 
course were provided with an R Markdown file with simulation code in order to 
demonstrate how the KS test fails with estimated parameters in the hypothesized 
distribution. The R Markdown file that was provided to students is included in 
the Supplemental Materials. This R-based applied linear regression course 
has the following prerequisites: two semesters of introductory statistics OR a 
Calculus~2-based statistical methods course.


At both the graduate and undergraduate level, students were able to follow the 
demonstrations closely and gain a clear understanding of the implications of 
using estimated parameters within test statistics.  At the graduate level, 
students further were able to appreciate the efficacy of the parametric 
bootstrap method as a corrective measure.



\subsection*{To Reviewer}


\begin{comment}
To address the first point, the paper relies on descriptive statistics, ranked
tests, and the results of a set of GLMMs. The descriptive data analysis lends
credence to the authors' argument that the reaction times at the 2022 World
Championships was faster than at any of the relevant comparisons. The rank test
appears to be set up correctly and I believe the authors are interpreting the
results correctly. Similarly, I believe the GLMM results are arrived at
correctly given the provided code.

The paper attempts to answer the second question by applying the GLMMs they
already fit. To do so, it presents simulations from the fitted GLMM. It then
uses the simulated results to estimate the probability of observing reaction
times below a set of thresholds. I think it does a good job of contextualizing
these in terms of the number of races per an occurrence.
\end{comment}


Thank you for these encouraging comments.  


\begin{comment}
However, I do believe the analysis to answer this first question could be
substantially improved. First, the paper only presents results for men running
the 110 meter hurdles. Women also competed at these events as they later
acknowledge in the discussion section. Why not use data from those competitors?
(There is a suggestion that the systematic effect seen at the 2022 World
Championships for men would not/did not appear for women because they may have
slower reaction times. No analysis is provided to support that though. Could
that be provided?) Similarly, other sprinting events were held using the same
block-sensor technology. Were similar patterns present in the 100-meter and
200-meter dashes? What about in the longer 400-meter dash and 400-meter hurdle
races? I think expanding the sample here would yield helpful insights into the
systematic nature of the problem the paper is trying to investigate. These
expanded samples could improve both the ranked-test results and the GLMM results
and improve the generalizability of the conclusions.
\end{comment}


One of the reasons why we did not include data from other sprinting evetns is
because we wanted to ground the article in Devon Allen's story and to be about
the hurdles events.  Athletes who compete in the longer events such as 400 meters
may not have as good of a start as those in the 100 meter events because reaction
time plays a much smaller role in the outcome of the race. While it is common
for amateur and collegiate athletes to compete in multiple sprinting events, it
seems reasonable that a world championship 100 meter runner may be only a top 20
200 meter runner.  We are trying to estimate the probabilty of observing a extreme
reaction time to determine whether 0.1 seconds is fair but if we include data
from competitions with slower reaction times, then 0.1 seconds is going to be
more extreme.


\begin{comment}
Relatedly, I believe the GLMM results could be improved by considering a few
more factors. Currently, the GLMMs regress reaction time onto models that only
contains predictors for year and whether or not an observation comes from a
preliminary heat or a final (both as random intercepts). It strikes me that the
authors may also want to account for confounding factors such as the overall
speed-level of the heat (especially relative to that point in time) and the
speed of the given runner relative to the others in the heat, as these factors
are likely to inform whether a runner tries to push their reaction to the limit
or not and are potentially correlated with year and race type, biasing the
results. Accounting for these additional features (in addition to expanding the
analysis) should help clarify if there was a specific difference attributable to
2022.
\end{comment} 


The heat effect measures the speed-level of the heat.  We added some additional
explanations to make it clear what the venue and heat effect refer to.


\begin{comment}
Otherwise, I think the manuscript could be improved by streamlining the
presentation of the organizing material like the problem statements and theory.
The literature review currently reads like an itemized list of all the relevant
studies; revising to focus on the specific arguments the paper hopes to make
could help improve clarity. As it is, I am not entirely sure if the paper is
specifically trying to test a hypothesis offered by others (e.g. the sensors
were calibrated in some way to lead to faster reaction times) or just present a
general framework for studying track and field down the road. I believe the
former is stronger and of more interest than the latter, so making clear the
purpose would help.

\end{comment}  
We do not have enough evidence to conclude that the sensors were calibrated
differently, so thus we make general recommendations for World Athletics.

\begin{comment}
Altogether, I see a lot of promising work here and think it would generally
benefit from expanded scope and analysis.
\end{comment}


Thank you for these encouraging words.

\subsection*{Reviewer2}


\begin{comment}
The paper’s topic is very important, but has significant issues with the writing
quality and statistical rigor. Right now, the manuscript reads more like report
than a paper. The text lacks significant direction, and does not explain the
goals of the study until the 4th page. Even then, “checking for abnormalities”
is very vague.

\end{comment}


We can restructure the paper but doesn't the abstract mention the goals of the
paper? I suppose the introduction is really more of a background section.


\begin{comment}
Concerns about statistical tests:

I am not convinced that the rank-based comparison test is the best choice here.
There are two questions: 1) Was the venue for the 2022 championships faster than
usual? 2) Was the final heat for the 2022 championships faster than usual?

If you are trying to answer question (1), you should include reaction times from
every spring event, not just the hurdles. If you are trying to answer question
2, why not do the following Monte Carlo test:

1. Sample a single reaction time for each athlete in the final.
2. Calculate the mean reaction time for the simulated final.
3. Compute the probability of seeing the observed average reaction time deviate
as much as it did in the 2022 Championships.

This test seems better than a rank sum test, as ranks do not capture the
differences between times.

\end{comment}

From what distribution are we sampling athlete's reaction times from?


\begin{comment}
I am also not convinced that the GLMM methods used for evaluating the
probability of disqualification are valid. The tests use data of reaction times
from performances that were NOT disqualified.  The data is therefore truncated
with no samples from the “left” or “disqualified” tail. This means that any
conclusions about the probability of a reaction time falling below this
threshold is purely extrapolation from the distributional assumptions. The
authors use a gamma distribution based on its fit to the data they have. In my
opinion, the choice of a gamma distribution is not sufficiently justified to
make extrapolating claims about the probability mass in the upsampled tail. I am
not convinced there is enough data to make a strong argument here, especially
when the previous literature on reaction time uses different distributions.
\end{comment}
Athletes may be disqualified based on PED use, and athletes may not finish a race
due to injury, thus we did not consider these times.  However, it does seem
reasonable to include times from people who were disqualified based on their
reaction time.

In terms of the choice of distribution, Brosnan and Harrison used a exponentially
modified gaussian distribution using athletes sex, ruling periods, competition
rounds. The article is now behind a pay wall.


\begin{comment}
The paper also mentions another claim that I think would be worth investigating:
Devon Allen’s reaction times might be faster due to innate ability. I think it
would be nice to show that his average reaction time faster, and that this
difference is statistically significant. It would also be nice to see how Devon
Allen’s DIFFERENCE in reaction time compares to the differences of the other
competitors (relative to previous heats, or averages, or whatever). If Devon
Allen’s time improved anomalously in the final heat, this would suggest a false
start. Based on the fact that he recorded a .101 reaction time earlier, I am
guessing he did not improve anomalously.  
\end{comment}

Do we want to perform some formal statistical analysis of Allen's times? What
data do we want to use? I don't think it makes sense to use Allen's times at
lower level competitions versus sprinters at the world championships.

\begin{comment}
A point that is not sufficiently emphasized is that the track officials consider
low reaction times to be false starts because an athlete can ANTICIPATE a start
faster than they can REACT to the starting gun. I think this is important to
discuss so that a general audience understands why an athlete might be
disqualified even through they started running after the race began.  
\end{comment}

This can be pretty easily addressed with a 1-2 sentence explanation in the
introduction.

\begin{comment}I have no idea how to interpret the numbers displayed in Table 2. These
need to be made interpretable, or they add nothing to the paper.
\end{comment}

Table 2 is the table with the parameters from the models.  This can be moved to
the appendix? We pretty explicity define all the terms in the methods section.

\begin{comment}
Line 127: Is there a reason you are commenting on the difficult of data
gathering? I am not sure it adds much.
\end{comment}

Can be removed.

\begin{comment}
Line 141: “We compare the times” - do you mean reaction times?
\end{comment}

Yes, this can be easily resolved.

\begin{comment}
Figure 3 is extremely inefficient. Maybe show the deltas on one plot? I'm not
sure a plot is even needed. What is the goal here? If you want to show that 2022
is anomalous you need to show where it stands in the distribution. Trying to
show that 2022 is anomalous based on a shift in the calculated venue effects
seems very roundabout.
\end{comment}

This was meant to be used as further evidence to show 2022 is an outlier. Perhaps
we delete this graph? I am unsure what the reviewer means by where "2022 stands
in the distribution"


\subsection*{Associate Editor}
\begin{comment}
I encourage you to consider the issues raised by the reviewers carefully. I have
little additional material to add, other than to emphasize the need to
re-organize the manuscript to more clearly identify the key goals of the work at
the start, and to provide a more concise and focused presentation of the key
findings.
\end{comment}

%\bibliographystyle{chicago}
%\bibliography{citations}
\end{document}
