\documentclass[12pt]{article}
\usepackage[margin=1in]{geometry}
\usepackage{natbib}
\usepackage{hyperref}

\usepackage{listings}
\usepackage{rotating, graphicx}
\graphicspath{{./}, {./image/}}
\usepackage{booktabs, natbib}
% \usepackage{breakurl}
% \usepackage [english]{babel}
\usepackage{amsmath, amsbsy, amsthm, epsfig, epsf, psfrag, graphicx,
amssymb, enumerate}
\usepackage{bm}
\usepackage{multirow, multicol}

\usepackage[dvipsnames]{xcolor}
\definecolor{darkblue}{rgb}{0.1, 0.2, 0.6}


\newcommand{\jy}[1]{\textcolor{orange}{JY: #1}}
\newcommand{\eds}[1]{\textcolor{blue}{(EDS: #1)}}
\newcommand{\of}[1]{\textcolor{purple}{(OF: #1)}}

\sloppy

% \usepackage{csquotes}
% \usepackage [autostyle, english = american]{csquotes}
% \MakeOuterQuote{"}

% \usepackage{bibentry}
\newenvironment{comment}%
{\begin{quotation}\noindent\small\it\color{darkblue}\ignorespaces%
}{\end{quotation}}


\begin{document}

\begin{center}
  {\Large\bf TAS-23-296: Response to the Comments}
\end{center}

We sincerely thank the Editor and Associate Editor for the opportunity
to revise our manuscript. The insightful comments and constructive
feedback from the reviewers have greatly enhanced the quality of this
work, and we are deeply grateful for their efforts.


The manuscript has been revised accordingly with the following most
notable changes:
\begin{enumerate}
\item
  The focus of the study has been clarified in the completely
  rewritten introduction. The two objectives are 1) to investigate
  whether the reaction times at the 2022 World Championships were
  significantly different from the other competitions; and 2) to
  assess the appropriateness of the 0.1-second threshold.
\item
  Additional data (reaction times from the 100-meter dash) have been
  added to the analysis for both objectives to expand the scope of the
  analysis and derive better insights.
\item
  In the assessment of the 0.1-second threshold, the Gamma GLMM has
  been replaced with a generalized Gamma distributions in the flexible
  framework of Generalized Additive Models for Location, Scale, and
  Shape (GAMLSS) \citep{stasinopoulos2024generalized}, with random
  effects of venue and heat in two of its parameters, respectively.
  The new model provides an adequate fit to the data, as confirmed by
  diagnostic checks.
\item
  Given the expanded dataset and change to GAMLSS, the corresponding
  sections of the paper, including data preparation, methods, results,
  and discussions have all been rewritten at large.
\end{enumerate}

Point-by-point responses to the comments are as follows, with the
comments quoted in \emph{\color{darkblue} italic and blue}.


\subsection*{To Associate Editor}

\begin{comment}
I encourage you to consider the issues raised by the reviewers carefully. I have
little additional material to add, other than to emphasize the need to
re-organize the manuscript to more clearly identify the key goals of the work at
the start, and to provide a more concise and focused presentation of the key
findings.
\end{comment}


Thank you for your feedback. As detailed in the summary of notable changes,
we have thoroughly addressed the reviewers' comments and restructured the
manuscript to clearly identify its key goals and present the main findings
concisely. We appreciate your time and consideration in re-evaluating our
submission.

\subsection*{To Reviewer 1}

\begin{comment}
Using Devon Allen's disqualification at the 2022 World Track and Field
championships as a hook, this paper poses two questions. First, were reaction
times at the 2022 World Track and Field Championships faster than they had been
earlier in 2022 national-level competitions, the 2019 World Track and Field
Championships, and the 2023 World Track and Field Championships? Second, based
on modeled reaction times, is the current standard of a 0.1 second threshold
for reaction times before disqualification too conservative? To
address the first point, the paper relies on descriptive statistics, ranked
tests, and the results of a set of GLMMs. The descriptive data analysis lends
credence to the authors' argument that the reaction times at the 2022 World
Championships was faster than at any of the relevant comparisons. The rank test
appears to be set up correctly and I believe the authors are interpreting the
results correctly. Similarly, I believe the GLMM results are arrived at
correctly given the provided code.

The paper attempts to answer the second question by applying the GLMMs they
already fit. To do so, it presents simulations from the fitted GLMM. It then
uses the simulated results to estimate the probability of observing reaction
times below a set of thresholds. I think it does a good job of contextualizing
these in terms of the number of races per an occurrence.
\end{comment}

We thank the reviewer for the accurate summary of our contributions
and for the encouraging feedback. As part of our revisions, we have
expanded the analysis to include reaction times from the 100-meter
dash. Additionally, we replaced the gamma GLMM with a generalized
Gamma distribution in the flexible framework of GAMLSS,  featuring
random effects in its parameters, which provides an improved and
adequate fit to the data.


\begin{comment}
However, I do believe the analysis to answer this first question could be
substantially improved. First, the paper only presents results for men running
the 110 meter hurdles. Women also competed at these events as they later
acknowledge in the discussion section. Why not use data from those competitors?
(There is a suggestion that the systematic effect seen at the 2022 World
Championships for men would not/did not appear for women because they may have
slower reaction times. No analysis is provided to support that though. Could
that be provided?) Similarly, other sprinting events were held using the same
block-sensor technology. Were similar patterns present in the 100-meter and
200-meter dashes? What about in the longer 400-meter dash and 400-meter hurdle
races? I think expanding the sample here would yield helpful insights into the
systematic nature of the problem the paper is trying to investigate. These
expanded samples could improve both the ranked-test results and the GLMM results
and improve the generalizability of the conclusions.
\end{comment}


Thank you for the valuable suggestions. In response, we expanded our
analysis to include reaction time data from both men and women's
100-meter dashes. A generalized gamma model fit to these data showed
no significant differences in reaction times by event type.


We also explored adding data from the 200-meter dash. However, a
generalized gamma model fit to the expanded dataset that included the
200-meter event revealed a significant effect of event type. Tukey’s
Honest Significant Difference (HSD) test found a p-value of 0.0021 for
the 200-meter dash compared to the 110-meter hurdles, and a p-value of
0.000017 for the 100-meter dash compared to the 200-meter dash.
These results indicate that reaction times in the 200-meter dash are
significantly slower. Consequently, we excluded the 200-meter event
from the main analysis, as its systematic differences in reaction
times could bias conclusions about extreme reaction times and the
appropriateness of the 0.1-second threshold.


The observed differences in reaction times between events are not
unexpected. Athletes competing in longer sprints, such as the
200-meter dash, may place less emphasis on reaction time compared to
those in shorter sprints like the 100-meter dash. Elite athletes at
the World Championship level often specialize in specific distances,
and performance across events may vary. Including data from events
with systematically slower reaction times would inflate the extremity
of a 0.1-second reaction time, potentially skewing the results of our
study, which focuses on the likelihood of extreme reaction times
relative to this threshold.


In the revision, we have added explanation about why 100-meter dash
data for both men and wemen were included, while 200-meter dash data
were not included in our analysis in the opening paragraph of Section~2.
The subsequent analyses in the manuscript have been updated with the
expanded datasets.


\begin{comment}
Relatedly, I believe the GLMM results could be improved by considering a few
more factors. Currently, the GLMMs regress reaction time onto models that only
contains predictors for year and whether or not an observation comes from a
preliminary heat or a final (both as random intercepts). It strikes me that the
authors may also want to account for confounding factors such as the overall
speed-level of the heat (especially relative to that point in time) and the
speed of the given runner relative to the others in the heat, as these factors
are likely to inform whether a runner tries to push their reaction to the limit
or not and are potentially correlated with year and race type, biasing the
results. Accounting for these additional features (in addition to expanding the
analysis) should help clarify if there was a specific difference attributable to
2022.
\end{comment}


Thank you for this thoughtful suggestion. We apologize for any
confusion regarding the heat effect. In our model, the heat effect
captures the speed-level of the heat rather than the type of heat
(e.g., preliminary vs. final). To clarify this, we have added
additional explanations Section~3.2 (after Equation (3)) to
explicitly define the venue and heat effects. We avoided calling it a
``race effect'' to prevent misinterpretation as referring to the
athlete's race.


We considered adding an effect for an individual runner’s average
ability; however, many observations in our dataset come from athletes
with only 1–-3 data points. Taking an average for such small samples
may not be informative, especially in the presence of outliers or
cases where an athlete had an atypical start. As such, we opted not to
include this factor in the model.


In the revision, we replaced the Gamma GLMM with a generalized Gamma
distribution in the GAMLSS framework. The venue effect and heat effect
were incorporated into two parameters controlling the scale of the
distribution. This revised model provides an improved and adequate fit
to the data compared to the original Gamma GLMM.


\begin{comment}
Otherwise, I think the manuscript could be improved by streamlining the
presentation of the organizing material like the problem statements and theory.
The literature review currently reads like an itemized list of all the relevant
studies; revising to focus on the specific arguments the paper hopes to make
could help improve clarity. As it is, I am not entirely sure if the paper is
specifically trying to test a hypothesis offered by others (e.g. the sensors
were calibrated in some way to lead to faster reaction times) or just present a
general framework for studying track and field down the road. I believe the
former is stronger and of more interest than the latter, so making clear the
purpose would help.
\end{comment}

We appreciate the feedback, which aligns with concerns raised by
Reviewer~2. To address this, we completely rewrote the introduction to
better structure our arguments and focus the problem statements. The
literature review was streamlined to highlight studies directly
relevant to our analysis. We clarified the paper’s two main
objectives: (1) identifying inconsistencies in reaction times at the
2022 World Championships, and (2) assessing the fairness of the
0.1-second threshold.


We avoided speculation about sensor calibration or implicating Seiko,
as there is no direct evidence to support such claims. Instead, we
provide general recommendations for World Athletics while noting that
the unusually fast reaction times observed in 2022 remain unexplained.


\begin{comment}
Altogether, I see a lot of promising work here and think it would generally
benefit from expanded scope and analysis.
\end{comment}

Thank you for the positive feedback. We have expanded both the rank
comparison and the GAMLSS model sections with additional data to
strengthen our analysis. These enhancements support our findings of an
anomaly in 2022 and highlight that the 0.1-second reaction time
threshold lacks a foundation in formal statistical evidence.


\subsection*{To Reviewer 2}

\begin{comment}
The paper’s topic is very important, but has significant issues with the writing
quality and statistical rigor. Right now, the manuscript reads more like report
than a paper. The text lacks significant direction, and does not explain the
goals of the study until the 4th page. Even then, “checking for abnormalities”
is very vague.
\end{comment}


Thank you for your valuable feedback. We have thoroughly revised the
Introduction and Discussion sections to clearly articulate the study's
objectives. As both reviewers noted organizational issues, we
streamlined the manuscript to focus on its two main goals: (1)
identifying inconsistencies in reaction times at the 2022 World
Championships, and (2) assessing whether the 0.1-second threshold is
statistically justified. Unrelated sections were removed to maintain a
clear and focused narrative.


Regarding the rank comparison and timing devices at the 2022
Championships, we intentionally used cautious language, as there is no
direct evidence to suggest miscalibration or faults in Seiko’s timing
devices. To avoid potential legal risks, we emphasized the observed
anomaly of unusually fast reaction times in 2022 without making
unsubstantiated claims. Instead, we presented the data transparently,
allowing readers to interpret the possible underlying causes of the
anomaly.


\begin{comment}
Concerns about statistical tests:

I am not convinced that the rank-based comparison test is the best choice here.
There are two questions: 1) Was the venue for the 2022 championships faster than
usual? 2) Was the final heat for the 2022 championships faster than usual?

If you are trying to answer question (1), you should include reaction times from
every spring event, not just the hurdles. If you are trying to answer
question~2, why not do the following Monte Carlo test:\\
\indent 1. Sample a single reaction time for each athlete in the final.\\
\indent 2. Calculate the mean reaction time for the simulated final.\\
\indent 3. Compute the probability of seeing the observed average reaction time deviate
as much as it did in the 2022 Championships.\\

This test seems better than a rank sum test, as ranks do not capture the
differences between times.
\end{comment}

Thank you for the feedback. We clarify that our focus is question~(1): 
evaluating whether the venue in 2022 was unusually fast. To address this, 
we expanded our analysis to include reaction times from additional sprint 
events at the World Championships, including both men’s and women’s 
100-meter dashes. This broader dataset provides a more comprehensive 
assessment of venue effects in 2022.


Your suggestion to include reaction times from more sprint events aligns 
with feedback from Reviewer~1.  While the 100-meter dash reaction times 
showed no significant differences from the 110-meter hurdles, we excluded 
200-meter and longer dashes due to their significantly different reaction 
times. These longer events represent distinct contexts unsuitable for this 
analysis.


\begin{comment}
I am also not convinced that the GLMM methods used for evaluating the
probability of disqualification are valid. The tests use data of reaction times
from performances that were NOT disqualified.  The data is therefore truncated
with no samples from the “left” or “disqualified” tail. This means that any
conclusions about the probability of a reaction time falling below this
threshold is purely extrapolation from the distributional assumptions. The
authors use a gamma distribution based on its fit to the data they have. In my
opinion, the choice of a gamma distribution is not sufficiently justified to
make extrapolating claims about the probability mass in the upsampled tail. I am
not convinced there is enough data to make a strong argument here, especially
when the previous literature on reaction time uses different distributions.
\end{comment}



Thank you for raising these concerns. Regarding the exclusion of 
disqualified runners, we chose not to include athletes with negative 
reaction times, as these represent attempts to predict the starting gun 
rather than true reactions. Negative reaction times do not align with 
our objective of evaluating the feasibility of the 0.1-second threshold, 
as they reflect pre-emptive behavior rather than genuine reaction
times.


The Gamma GLMM model has been replaced with a generalized Gamma 
distribution within the GAMLSS framework, incorporating random effects 
at the venue and heat levels in its distributional parameters. This 
revised model provides improved and adequate fit to the data, as shown 
by diagnostic plots in Figure~3.


For the approximately 20 disqualified athletes with positive reaction
times, we examined the impact of including them in the analysis. The
probability of observing a reaction time below 0.1 seconds increased
\jy{Are these based on the expanded dataset or the original dataset?}
\jy{These numbers need to be updated}
\of{Used 2 datasets: both men's dash and hurdles with 2022.  One dataset did not
include DQs the other did. The numbers reflect the supplement results}
slightly from 0.00197 to 0.00276, with similar parameter estimates for
the random effects. Given these results, we opted to include these
observations in the dataset.
\eds{Currently the text in the manuscript suggests we are including a 
sensitivity analysis in the Supplement for the analysis without the positive DQs.
If we are indeed including that, we should mention that here too. (If not, we 
need to remove that statement from the main manuscript.)}

\begin{comment}
The paper also mentions another claim that I think would be worth investigating:
Devon Allen’s reaction times might be faster due to innate ability. I think it
would be nice to show that his average reaction time faster, and that this
difference is statistically significant. It would also be nice to see how Devon
Allen’s DIFFERENCE in reaction time compares to the differences of the other
competitors (relative to previous heats, or averages, or whatever). If Devon
Allen’s time improved anomalously in the final heat, this would suggest a false
start. Based on the fact that he recorded a .101 reaction time earlier, I am
guessing he did not improve anomalously.
\end{comment}


Thank you for this suggestion. Unfortunately, comprehensive data on 
Devon Allen’s reaction times is not readily available. While we gathered 
as much data as possible from publicly available sources, reaction times 
are typically recorded only at major competitions, such as national and 
world championships. As a result, the dataset is too limited to support 
a formal statistical analysis of his reaction times.
To address this point, we have added a brief discussion in the first 
paragraph of the Discussion Section, noting the limitations in data 
availability and the implications of this issue.


\begin{comment}
A point that is not sufficiently emphasized is that the track officials consider
low reaction times to be false starts because an athlete can ANTICIPATE a start
faster than they can REACT to the starting gun. I think this is important to
discuss so that a general audience understands why an athlete might be
disqualified even through they started running after the race began.
\end{comment}

We added additional explanations in the second and third paragraph of the
introduction to address this.


\begin{comment}
I have no idea how to interpret the numbers displayed in Table 2. These
need to be made interpretable, or they add nothing to the paper.
\end{comment}

Thank you for pointing out the need for clearer interpretation of
Table~2. The table now contains fitted parameters of the generalized Gamma
distribution GAMLSS as specified Equations (1)--(3). The table caption has been
expanded to provide detailed explanations of the fitted parameters.


\begin{comment}
Line 127: Is there a reason you are commenting on the difficult of data
gathering? I am not sure it adds much.
\end{comment}


Yes, the comments about the difficulty of data gathering are included
to highlight the effort required to compile a comprehensive
dataset. This also serves as a note to World Athletics, emphasizing
the need for better data standardization. Currently, race results on
their website do not include reaction times, requiring us to search
alternative sources. Gathering data from national-level meets in 2022
involved searching each country’s championship results individually,
often encountering language barriers that complicated interpretation.


\begin{comment}
Line 141: “We compare the times” - do you mean reaction times?
\end{comment}

Yes, this has been rephrased for clarification.

\begin{comment}
Figure 3 is extremely inefficient. Maybe show the deltas on one plot? I'm not
sure a plot is even needed. What is the goal here? If you want to show that 2022
is anomalous you need to show where it stands in the distribution. Trying to
show that 2022 is anomalous based on a shift in the calculated venue effects
seems very roundabout.
\end{comment}


Given the confusion it may cause, this figure has been removed.

% Thank you for the feedback. The purpose of Figure~3 is to display the venue 
% effects for each year and visually illustrate how 2022 compares to other years. 
% To address your concern, we added additional comments quantifying the 
% probability of observing an extreme venue effect, which strengthens the 
% argument that 2022 is anomalous. If needed, we are open to revising the plot 
% further or integrating it into a single panel showing deltas, as suggested.
% \eds{let's revisit this again after objective 2 analyses are set}


\bibliographystyle{apalike}
\bibliography{citations}
\end{document}
